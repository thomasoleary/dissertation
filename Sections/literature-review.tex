\section{Literature Review}
My literature review consists of two parts that I believe to be important to my research question. It first describes different implementations of Procedural Interior Generation (PCIG) then explores ways in which Artificial Intelligence (AI) is compared to Humans.

\subsection{Implementations of Procedural Interior Generation}
Although PCG has a lot to show and offer in game development - being used for characters, terrain, weapons and textures - the use of Procedural Interior Generation (PCIG) in games however is scarcely come by.
\\
A game that does use PCIG is Catlateral Damage \cite{game:catlateral},
a small indie game developed by Manekoware where you play as a cat on a destructive rampage 
in its own house. In 2017, Chris Chung (the indie developer behind the game) wrote a case study about the level design in his game \cite{what-is-pcg}. When developing Catlateral Damage, Chung was undecided on how to design the levels and ultimately went for PCIG\cite{pcg_in_gd}.
Before the interior decoration can take place, a Squarified Treemap algorithm is used to generate the room layouts and floor plans within the level \cite{squarified-treemap}. Each room generated from this algorithm has an associated data file, this file contains the type of furniture available, maximum type of each furniture, spawn probabilities, if the furniture is placeable on walls and if the player will spawn there. The furniture objects that can be placed have physics components attached, to allow them to be accurately placed within the level - for this, a Rectangle Packing algorithm is used to place these objects within allocated surface areas on the floor and other furniture objects. Concluding the case study, Chung states that most players could not notice that the levels were procedurally generated -
\textbf{although this is a promising statement, Chung has not shown any evidence to back this claim.}
\\
\\
Despite there not being many implementations of PCIG in games, there are however a handful of published papers that have used their own techniques to emulate room interiors. \\
\subsubsection*{Multi-Agent System}
In 2009, T. Germer and M.Schwarz sought out to procedurally arrange a rooms' furniture in real-time. \cite{real-time-walkthroughs}. They did this by producing a Multi-Agent based solution where each furniture object, in a given room, is seen as an individual agent that seeks a suitable parent furniture object.
The tool has 3 main requirements: \cite{real-time-walkthroughs}
\begin{itemize}
    \item Generate furniture arrangements rapidly
    \item Arrangements must be persistent
    \item Must be plausible and interesting
\end{itemize}
As stated earlier, each agent seeks a suitable parent in a given room - once found it would place and orient itself accordingly. These agents have custom semantic descriptions to allow them to create different object layouts, an example listed by the authors is a chair - a chair can either be set next to a table/desk but can also be isolated in its own surroundings leading it to have many possible parent objects \cite{real-time-walkthroughs}.\\
Each agent has only 3 states:
\begin{enumerate}
    \item Search
        \begin{itemize}
            \item All agents start in the \textit{Search} state, they begin by searching for possible parent objects - if a parent is found that suits its semantics the agents' state changes to \textit{Arrange}, if a parent can't be found at all the agent is deleted.
        \end{itemize}
    \item Arrange
        \begin{itemize}
            \item In the \textit{Arrange} state, the agent attempts to place and orient itself with the parent accordingly. Whilst doing so, it has to check for collisions with other agents in which the Separating Axis Thereom is used \cite{separating-axis-thereom} - if no collisions are found the agents' state changes to \textit{Rest}.
        \end{itemize}
    \item Rest
        \begin{itemize}
            \item In the \textit{Rest} state, potential child agents are now able to seek this object as a parent. If the resting agents parent moves, the resting agent will move along with it - however if this move results in a collision, its parent is lost and the agents state is changed back to \textit{Search}.
        \end{itemize}
\end{enumerate}


YouTube video demonstrating the implementation \cite{youtube:real-time-walkthroughs}
\subsubsection*{Rule-Based Layout}
Paper \cite{rule-based-layout}
\subsubsection*{Statistical Relationships}
Paper \cite{make-it-home}
\subsubsection*{Constraints}
Paper: \cite{constrained-layouts}, uses SUNCG dataset \cite{suncg}

\bigskip
On 29th April 2021, Sony Interactive Entertainment published patent US20210121781, titled "AI-Generated Internal Environments Based On External Geometry" \cite{sony-patent}. The patents' description goes onto explain a Machine Learning (ML) tool that takes in data from the external structure of a virtual building and generates an interior environment just from this data. Although this is just a patent for an ML tool, this could be the start of PCIG being used in AAA Titles. 

\subsection{Artificial Intelligence Compared to Humans}
This is going to be a little more difficult to right about, as I haven't read a paper on this so far. And I have only managed to find 3 papers that talk about this, but I am not sure that they could be entirely relevant.