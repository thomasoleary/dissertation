\section{Literature Review}
My literature review consists of two parts that I believe to be important to my research question.
It first describes different implementations of Procedural Interior Generation (PCIG) then explores
ways in which Artificial Intelligence (AI) is compared to Humans.

\subsection{Implementations of Procedural Interior Generation}
Although PCG, as a whole, has a lot to show and offer in game development - occasionally being used for
characters, terrain, weapons and textures - the use of specifically Procedural Interior Generation 
(PCIG) in games are scarcely heard  of.
A game that does use PCIG however is Catlateral Damage Remeowstered\cite{game:catlateral}.
A small indie game developed by Manekoware where you play as a cat on a destructive rampage in its own house.
When developing Catlateral Damage, developer Chris Chung was decided on how to design the levels and ultimately decided to use PCG and PCIG\cite{pcg_in_gd}.
Before the interior decoration can take place, a Squarified Treemap algorithm is used to generate the room layouts and floor plans within the level.

\bigskip
As of 29th of April 2021, Sony Interactive Entertainment own patent US20210121781, 
titled "AI-Generated Internal Environments Based On External Geometry" \cite{sony-patent}.
The patents' description goes onto explain a 
Machine Learning (ML) tool that takes in data from the external structure of a virtual building
and generates an interior structure just from this data.
Although this is just a patent for an ML tool, this could be the start of PCIG being used in AAA Titles. 

\subsubsection*{Multi-Agent System}
Paper \cite{real-time-walkthroughs}
\subsubsection*{Rule-Based Layout}

\subsubsection*{Statistical Relationships}


\subsection{Artificial Intelligence Compared to Humans}
This is going to be a little more difficult to right about, as I haven't read a paper on this so far.
And I have only managed to find 3 papers that talk about this, but I am not sure that they could be entirely relevant.