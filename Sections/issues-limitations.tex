\section{Issues \& Limitations}\label{issues}
\subsection{Viewing format in Study}
Although no points were made during the Pilot Study of this research regarding the viewing format of the room interiors - many participants during the research study verbally stated that they found the viewing format ``off-putting''. By this they were referring to the way the rooms were represented on screen (See Fig~\ref{pair-example-1}). Some believed that they couldn't see enough of the room to justify picking one or the other and ultimately struggled in deciding what rooms to pick in both stages of the study.
\\
Mentioned earlier in justifying my experimental design, previous works by \italic{P. Henderson, et al.} \cite{constrained-layouts} and \italic{L.-F. Yu et al.} \cite{make-it-home} also carried out perceptual studies requiring participants to pick between human designed interiors and those designed from their own models.
\\
In the perceptual study ran by \italic{P. Henderson, et al.} \cite{constrained-layouts} (Constraint based approach), multiple viewing formats were applied to represent the room interiors depending on what type of room was generated. When displaying unconstrained room types to participants, both overhead and first person view formats were used. Regarding results for these layouts, an overhead viewing format was seen to be almost identical to human and a first person viewing format was slightly preferred.
\\
However, in the perceptual study ran by \italic{L.-F. Yu et al.} \cite{make-it-home} (Statistical approach) 3 viewing formats were used to represent the layouts - an overhead and 2 different views from the corners of the room. The results showed that in 3 of the paired comparisons, the human designs were not clearly preferred over the synthesized arrangements.
\\
Perhaps by showing multiple viewing formats of both Artefact and Human designed interiors - it may have made for a much more comfortable experience for some participants, and we may have seen different choices made in both stages of the study.

\subsection{Participants}
Out of the 56 participants involved in the research study, 53 of these were students from the Games Academy (Falmouth University). With a good sense of game development/design and a general census of what is achieved by others within the Games Academy, some participants may have been able to predict the involvement of a non-human element to the research study before being notified of the Artefact in Stage 2. This possible prediction skews the expectation of a participant not having prior knowledge of the Artefact's involvement, possibly further skewing results.
To help with this, it would have possibly been more beneficial for the study to have a wider range of participant knowledge surrounding digital games and Artificial Intelligence (AI) - branching outside the Games Academy for participants. But due to the strict time schedule I had set myself for this research project, asking students within the Games Academy to be participants in order to reach my sample size was the easiest option available.