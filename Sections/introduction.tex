\section{Introduction}

Urban open world games such as Grand Theft Auto V \cite{game:gta}, The Division \cite{game:division} 
and Batman: Arkham Knight \cite{game:arkham-knight}
have such large built-up areas for players to venture in.
However, only a few handpicked buildings in these large cities are accessible and have 
modelled interiors leaving the others to be blocked off for decorative purposes.
This could be resolved by modelling and designing each room in these cities, 
but this would become incredibly impractical. 
Other issues with this can lean towards rendering and the storage of such heavily dense areas.

\subsection*{Procedural Generation}
Procedural Generation (PCG) refers to automatically creating content using algorithms \cite{what-is-pcg}.
PCG has many applications in video games,
some notorious examples being the world/cave generation in Minecraft \cite{game:minecraft},
the character, civilization and world generation in Spore \cite{game:spore}
and the procedural texture and music generation in .kkrieger \cite{game:kkreiger}.

Using PCG, this largely time-consuming task of designing room interiors can be automated. 
\textbf{And can possibly help maintain a player's immersion within the game. }
An issue with this however is that PCG tool's can be seen as boring and repetitive \cite{pcg_in_gd}.

Through my literature review though I have found many implementations and techniques of Procedural Interior 
Generation (PCIG), none of these get compared to Human designed interiors. 
\bigskip

This study looks to see if a participant is able to tell the difference between Human designed and AI 
generated interiors.

