\section{Discussion}
Regardless of the upsetting results of both Hypotheses failing to reject the null, the information collected from my research study can still prove to be of some use as a preliminary test to help towards future studies regarding comparisons between Procedural Interior Generation and Human design.
\\
\\
Looking at Fig~\ref{stage-1-graph}, we can see that Artefact interiors are chosen more than Human interiors. In this stage, the Artefact was selected 147 times whereas the Human interiors were selected 133. Although not statistically significant, this does show some evidence to say that the artefact was preferred over human designs in the research study.
\\
\\
When looking at Fig~\ref{stage-2-graph}, we can quite clearly see that participants selected Artefact interiors a lot more than human. In this stage, the Artefact was selected 154 times whereas the human rooms were selected 126. If \italic{Hypothesis 2} were to be changed to that of \italic{"When notified, the participant IS able to distinguish between Human and Artefact (M-A System) interiors"}, this would have lead to a lower-tailed Z-Test checking if the Human designs were picked less than Artefact designs. Resulting in a Z-Score of 1.67 and a P value of 0.0475, at P $<$ 0.05 the null hypothesis would have been rejected thus accepting the alternate.