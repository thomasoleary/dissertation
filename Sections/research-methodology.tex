\section{Research Methodology}

\subsection{Experimental Design}
In laymen terms my research study will involve an A/B test similar to the work from \textit{P. Henderson, et al.} \cite{constrained-layouts} and \textit{L.-F. Yu et al.} \cite{make-it-home}.
\subsection{Limitations}
Time, resources

\subsection{Sampling Plan}
Sample size, sampling method

\subsection{Data management plan}
Managing, collecting, \& storing data

\subsection{Data Analysis}
See \hyperref[append:b]{Appendix B}

\subsection{Ethical Considerations}
As the research study requires human participants this creates a medium ethics risk according to the Falmouth University Ethics Board. To facilitate this risk, a Falmouth University ethics form has been completed and will be signed off by the project Supervisor in consultation with the Head of Subject. The artefact itself is of low risk/concern as it will not be used for militarization and participants are able to opt out at any point during their participation.
Due to the nature of this study, no personal information about the participants involved will be collected - the European Union's General Data Protection Regulation (GDPR)\cite{gdpr} does not need to be followed. However, to protect the participants rights, the Nuremberg Code will be followed to keep and ensure this research study is ethically sound\cite{nuremberg-code}. All participants will be handed a Participant Information Sheet that details the key information they must know before the study, a consent form is also supplied to ensure they have agreed to participate. Participants are still able to withdraw at any point until submitting data.
\subsubsection*{COVID-19}
At the current state of the pandemic and following the latest Government Guidelines in England \cite{gov-guidlines}, participants will be requested to wear a face covering during their participation. All surfaces will be sanitised between usages.