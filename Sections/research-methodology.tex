\section{Research Methodology}

% NEED TO REWRITE THIS SECTION

\subsection{Experimental Design}
A participant will be shown 10 pairs of room layouts in the form of an A/B test. The participant will be informed to pick (\italic{out of each pair}) which one they prefer.
This preference may refer to realism, authenticity or what environment they'd rather be in. In each pair, one layout is human designed and the other will be designed by the artefact (M-A system). The order in which these layouts are shown in their pairs will be randomised.

After this initial stage is done, the participant will be informed that in every pair, one layout is generated by an Artificial Intelligence (the M-A system). Their challenge now is to identify, out of the same 10 pairs, which layouts are generated by the artefact.

This experimental design was been proposed as previous works by \italic{P. Henderson, et al.} \cite{constrained-layouts} and \italic{L.-F. Yu et al.} \cite{make-it-home} have carried out similar perceptual studies requiring a participant to pick between one from their designed model and human designs. In 2010, Margaret Boden proposed a new variant of the Turing Test (TT) that is oriented around artistic creativity \cite{artistic-tt}. With her TT, for an art program to pass it would have to:
\begin{enumerate}
    \item Be indistinguishable from human produced artwork
    \item[]And/Or
    \item Be seen having similar aesthetic value to human produced artwork
\end{enumerate}
I found this variant of the TT as a helpful source when deciding what my experimental design should be.

\subsection{Sampling Plan}
To help answer my Hypothesis, I will be using two tailed T-Tests to allow me to easily identify the relationship between the selection of Human and Artefact designed layouts.
Using G*Power \cite{gpower}, I was able to calculate a sample size of 54 with an effect size of 0.5. 

\subsection{Data management plan}
As explained in Section \hyperref[ethics]{E}, no personal information will be collected during the study - signifying that General Data Protection Regulation (GDPR) \cite{gdpr} does not need to be followed. Results collected from the study however will be exported and stored as a CSV file. This file will be password encrypted to prevent any third party intervention.

\subsection{Data Analysis}
Once I have sufficient collected data, I will use R-Studio for my analysis. In the code sample listed in \hyperref[append:b]{Appendix B}, I have demonstrated how to complete a Two Tailed T-Test using data from an imported CSV file.

\subsection{Ethical Considerations}\label{ethics}
As the research study requires human participants this creates a medium ethics risk according to the Falmouth University Ethics Board. To facilitate this risk, a Falmouth University ethics form has been completed and will be signed off by the project Supervisor in consultation with the Head of Subject. The artefact itself is of low risk/concern as it will not be used for militarization and participants are able to opt out at any point during their participation.
Due to the nature of this study, no personal information about the participants involved will be collected - the EU's General Data Protection Regulation (GDPR)\cite{gdpr} does not need to be followed. However, to protect the participants rights, the Nuremberg Code will be followed to keep and ensure this research study is ethically sound\cite{nuremberg-code}. All participants will be handed a Participant Information Sheet that details the key information they must know before the study, a consent form is also supplied to ensure they have agreed to participate. Participants are still able to withdraw at any point until submitting data.
\subsubsection*{COVID-19}
At the current state of the pandemic and following the latest Government Guidelines in England \cite{gov-guidlines}, participants will be requested to wear a face covering during their participation. All surfaces will be sanitised between usages.