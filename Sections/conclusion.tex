\section{Conclusion \& Future Work}

As discussed in \hyperref[issues]{Section VII}, particular elements of the research study can be adjusted to possibly administrate more beneficial data in whether participants can distinguish between Multi-Agent designed and Human designed Interiors.
\\
\\
Inspired by how \italic{P. Henderson, et al.} \cite{constrained-layouts} and \italic{L.-F. Yu et al.} \cite{make-it-home} handle it in their work - implementing different viewing frustums for the layouts in the experimental design, could potentially open an alley for more thoughtful decisions made by participants as they have more information to take in regarding each room layout in both stages of the study. It could allow them to analyse the room in more depth and perhaps visualise the room with much more ease with the multiple perspectives.
\\
\\
Another aspect of the study that could be adjusted is the selection of participants. 
expanding participant selection
gaining a larger pool of participants
more random sample
\\
\\
conclusion
sought to answer "Can an Unknowing Participant distinguish between Multi-Agent Designed and Human Designed Interiors?"
